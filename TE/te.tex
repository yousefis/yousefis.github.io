%\documentstyle[11pt,a4]{article}
%\documentclass[a4paper]{article}
\documentclass[a4paper, 10pt]{article}
% Seems like it does not support 9pt and less. Anyways I should stick to 10pt.
%\documentclass[a4paper, 9pt]{article}
\topmargin-2.0cm

\usepackage{fancyhdr}
\usepackage{pagecounting}
\usepackage[dvips]{color}
\usepackage{wrapfig}
\usepackage{graphicx}
\usepackage{hyperref}

% Color Information from - http://www-h.eng.cam.ac.uk/help/tpl/textprocessing/latex_advanced/node13.html

% NEW COMMAND
% marginsize{left}{right}{top}{bottom}:
%\marginsize{3cm}{2cm}{1cm}{1cm}
%\marginsize{0.85in}{0.85in}{0.625in}{0.625in}

\def\ie{{\em i.e.,}}
\def\eg{{\em e.g.,}}
\newcommand{\tabref}[1]{Table~\ref{#1}}
\newcommand{\figref}[1]{Figure~\ref{#1}}
\newcommand{\secref}[1]{Section~\ref{#1}}

\advance\oddsidemargin-0.65in
%\advance\evensidemargin-1.5cm
\textheight9.2in
\textwidth6.75in
\newcommand\bb[1]{\mbox{\em #1}}
\def\baselinestretch{1.05}
%\pagestyle{empty}

\newcommand{\hsp}{\hspace*{\parindent}}
\definecolor{gray}{rgb}{0.4,0.4,0.4}
%\definecolor{gray}{rgb}{1.0,1.0,1.0}


\begin{document}
\thispagestyle{fancy}
%\pagenumbering{gobble}
%\fancyhead[location]{text} 
% Leave Left and Right Header empty.
\lhead{}
\rhead{}
%\rhead{\thepage}
\renewcommand{\headrulewidth}{0pt} 
\renewcommand{\footrulewidth}{0pt} 
\fancyfoot[C]{\footnotesize \textcolor{gray}{\href{https://norouzi.github.io/}{https://norouzi.github.io/}}} 

%\pagestyle{myheadings}
%\markboth{Sundar Iyer}{Sundar Iyer}

\pagestyle{fancy}
\lhead{\textcolor{gray}{\it Mohammad Norouzi}}
%\rhead{\textcolor{gray}{\thepage/\totalpages{}}}
%\rhead{\thepage}
%\renewcommand{\headrulewidth}{0pt} 
%\renewcommand{\footrulewidth}{0pt} 
%\fancyfoot[C]{\footnotesize http://www.stanford.edu/$\sim$sundaes/application} 
%\ref{TotPages}

% This kind of makes 10pt to 9 pt.
%\begin{small}

%\vspace*{0.1cm}
\begin{center}
{\LARGE \bf Teaching Statement}\\
\vspace*{0.1cm}
{\normalsize Mohammad Norouzi (mnorouzi@google.com)}
\vspace*{0.2cm}
\end{center}
%\vspace*{0.2cm}

%\begin{document}
%\centerline {\Large \bf Research Statement for Mohammad Norouzi}
%\vspace{0.5cm}

% Write about research interests...
%\footnotemark
%\footnotetext{Check This}

\vspace*{-.1cm}
I am passionate about teaching and mentorship in the academic setting,
and this is the {\em main} reason that I am interested in an academic
position, as opposed to my current research scientist role at Google
Brain. I am satisfied with my current position, my open-ended research
adgenda, and my research collaborations.  However, I miss the
impactful long-term interactions with undergraduate and graduate
students, and the strive for finding the simplest and most effective
way to teach the students about machine learning and algorithms.

I think of teaching as an {\em opportunity} for the teacher to find
the most coherent and convincing story that connects the dots in the
mind of the students. Through this process of distilling knowledge,
the teacher and students develop new insights into existing
literature, which can often lead to new research ideas.  In addition,
through teaching, the teacher and the students form a special bond,
which can last forever. This special human connection is valuable to
me, and I feel honored when an old student remembers me after many
years and tells me about their positive learning experience. I
certainly remember my own teachers from elementry to graduate school,
and I am grateful for their lasting impact on the trajectory of my
life and career.

My first teaching experience as the sole lecturer was during
undergraduate school, when I had weekly classes at my own high school
teaching problem solving and the basics of programming. Because of my
background in Informatics Olympiad, I was invited to teach data
strcuture and algorithms to the members of the Iranian team in
Informatics Olympiad, to help prepare them for international
competitions. At the University of Toronto, I have served as a
teaching assistant on $7$ courses, taught tutorial sessions and held
office hours in ``Foundations of Computer Vision'', ``Machine Learning
and Data Mining'', ``Algorithm Design, Analysis and Complexity'',
``Data Structures and Analysis'', and ``Introduction to Theory of
Computation''. I have served as a guest lecturer at UC Berkeley on $3$
occasions in graduate courses on ``Deep Reinforcement Learning'' and
``Special Topics in Deep Learning''. I have given invited talks at
Machine Learning and Computer Vision conferences, different
universities, and research labs. I gave all of these talks and
lectures enthusiastically, which made me realize that I have a
long-term interest in teaching.

I am passonate about teaching a range of courses related to my
research background including ``Machine Learning'', ``Deep Learning'',
``Reinforcement Learning'', ``Machine Learning for Vision'', and
``Machine Learning for Natural Language Processing''. In addition, I
am interested in and capable of teaching basic Computer Science
courses such as ``Data Structures'', ``Algorithms'', and ``Programming
Languages''. Moreover, I intend to create the curriculum and teach the
following advanced courses:
\vspace*{-.1cm}
\begin{itemize}
\item {\bf Deep Generative Models.} There has been a surge of
  interesting research development in this area that deserve an
  independent course. This course will cover autoregressive models,
  variational auto-encoders, generative adversarial networks, and
  models based on normalized flow. I will also discuss traditional
  structured prediction models, and the new generation of sequence to
  sequence models with corresponding discriminative objective functions.
\vspace*{-.1cm}
\item {\bf Special Topics in Reinforcement Learning.} There has been a
  surge of interesting research on deep reinforcement learning
  algorithms in the last few years. However, given the complexity of
  the topic, the students often get lost between different
  formulations with different assumptions and incompatible
  experiments.  This course aims to select the most infulential recent
  papers in Reinforcement Learning and categorize them to facilitate
  an in-depth and coherent study.
\end{itemize}

During my time at the University of Toronto, I mentored ``Ali
Punjani'' (an undergraduate student at the time) and, we published two
papers together. During my time at Google, I have mentored $3$ interns
and over $10$ Google Brain residents. My collaborations with interns
and residents have been particularly fruitful, and the first author of
most of my recent publications is either an intern or a brain
resident. It has been particularly rewarding to see my mentees grow,
turn into successful researchers, and get hired by our team or move
onto excellent PhD programs. A few of my successful mentees include:
``Ofir Nachum'', ``Azalia Mirhoseini'', ``Irwan Bello'', ``Hieu
Pham'', ``Yun Liu'', ``Chen Liang'', and ``Supasorn Suwajanakorn''. I
believe in collaborative research and I have worked closely with a
long list of researchers on the Brain team during my $3$ years here.
Different researchers bring complementary ideas and different
perspectives into the table and help evolve research ideas. I believe
that universities play a key role in training critical thinkers and
successful researchers, and I cannot wait to take part in this
endeavor.

\vspace{0.5cm}
%\begin{flushright}
%Mohammad Norouzi
%\end{flushright}

%\end{small}
%\newpage

%\begin{thebibliography}{deSolaPITH}
% Change font size?
% \tiny, \footnotesize, \small,\normalsize, \large, \Large, \LARGE, and \huge 
%\begin{small}
\begin{small}

\bibliographystyle{plain}
\bibliography{bib.bib}

\end{small}

\end{document}

